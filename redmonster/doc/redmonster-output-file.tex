\documentclass[12pt]{article}
\parindent=0pt
\parskip=8pt
\begin{document}

\title{Documentation of Redmonster Output File}

\author{Timothy A. Hutchinson}

\maketitle

\section{Introduction}

This document describes the format of the \textit{redmonster} file,
which contains the spectroscopic redshifts and classifications of
the \texttt{redmonster} software.

\section{File Format}

\subsection{File naming convention}

Files from \texttt{redmonster} will generally follow the naming scheme

\hspace*{36pt}\textit{redmonster-pppp-mmmmm.fits}

where \textit{pppp} is the 4-digit SDSS plate number, and \textit{mmmmm}
is the 5-digit MJD.  These correspond with the \textit{spPlate} input file and
and cannot be changed by the user.

In the case where a file with the above naming convention exists in the given path, 
the default behavior of the software is to overwrite it.  However, the user may 
elect to leave the older file intact, in which case the new file will be written as

\hspace*{36pt}\textit{redmonster-pppp-mmmmm-YYYY-nn-dd\_HH:MM:SS.fits}

where \textit{YYYY}, \textit{nn}, \textit{dd}, \textit{HH}, \textit{MM}, and \textit{SS}
are the year, month, day, hour, minute, and second, respectively, of the time at which
the file was written.

\subsection{File type}

All \texttt{redmonster} outputs are uncompressed FITS
files with all relevant information in the primary HDU header and the first and second
BIN tables.  The file size is approximately 20 Mb.

\subsection{File contents}

The general structure of the file is as follows:

\begin{center}
	\begin{tabular}{ | l | l | l |}
	\hline
	HDU0 & NULL & Empty \\ \hline
	HDU1 & Binary FITS Table & Object redshifts and classifications \\ \hline
	HDU2 & NFIBERS x NPIX float image & Best-fit template model for each object \\
	\hline
	\end{tabular}
\end{center}

The primary HDU header is identical to that of the \textit{spPlate} files.
The header keywords are as follows:

\begin{center}
	\begin{tabular}{ | l | l | }
	\hline
	SIMPLE & FITS STANDARD \\ \hline
	BITPIX & PIXEL \\ \hline
	NAXIS & NUMBER OF AXES \\ \hline
	EXTEND & \\ \hline
	TAI & 1st row - Number of seconds since 17 Nov 1858 \\ \hline
	RA & 1st row - Right ascension of telescope boresight \\ \hline
	DEC & 1st row - Declination of telescope boresight \\ \hline
	EQUINOX & \\ \hline
	RADECSYS & \\ \hline
	AZ & 1st row - Azimuth of telescope \\
	\hline
	\end{tabular}
\end{center}


\subsection{Header structure and requirements}

The following primary header keywords are \textbf{required} to
be present and defined as specified, in addition
to header keywords required by the FITS standard itself:

\noindent \texttt{CRPIX1}: Shall be set to the value 1, referencing the
first sample point (``pixel'') along the wavelength axis (one-based indexing).

\noindent \texttt{CRVAL1}: Shall specify the
\textbf{base-10 logarithm} of the central wavelength
\textbf{in vacuum Angstroms} of first pixel along the wavelength axis.

\noindent \texttt{CDELT1}: Shall specify the pixel-to-pixel
increment in \textbf{base-10 logarithm} of vacuum wavelength from
one pixel to the next along the wavelength axis. \\


The following primary header keyword is \textbf{required} in the
case that the physical normalization of the templates is meaningful:

\noindent \texttt{BUNIT}: String giving the units of the template spectra. \\

The following primary header keyword is \textbf{required} in the
case that the templates are given with respect to air wavelengths
rather than vacuum:

\noindent \texttt{AIRORVAC}: String that is either \texttt{'air'}
or \texttt{'vac'} depending upon wavelength convention.
(The only significant value is \texttt{'air'}, since
anything else, including the absence of this keyword, will
be interpreted as \texttt{'vac'}.) \\

The following primary header keywords are supported as
optional but \textbf{recommended}
in some combination as appropriate:

\noindent \texttt{CNAMEn}: For $n > 1$, a string giving the
name of the physical parameter coordinate along the $n^{\mathrm{th}}$ axis.

\noindent \texttt{CUNITn}: For $n > 1$, a string giving the
units of the physical parameter coordinate along the $n^{\mathrm{th}}$ axis.

\noindent \texttt{CRPIXn, CRVALn, CDELTn}: For $n > 1$, specifying
the physical parameter baselines for axes corresponding to
regularly gridded numerical physical parameters.

\noindent \texttt{PVn\_j}: For $n > 1$ and for the case of irregularly gridded
numerical physical parameters along axis $n$ , specifies the parameter value
of the $j^{\mathrm{th}}$ point along the $n^{\mathrm{th}}$ axis.
The index $j$ shall begin with 1 and increase in integer steps up to the
size of the $n^{\mathrm{th}}$ axis, with one keyword per grid step.
Note that the FITS standard currently limits the maximum size of axes
that can be represented in this manner to 99, and that leading zero-padding
is not allowed.

\noindent \texttt{PSn\_j}: For $n > 1$ and for the case of non-numerical
physical parameters along axis $n$ , specifies the parameter string value
of the $j^{\mathrm{th}}$ point along the $n^{\mathrm{th}}$ axis.
The index $j$ shall begin with 1 and increase in integer steps up to the
size of the $n^{\mathrm{th}}$ axis, with one keyword per grid step.
Note that the FITS standard currently limits the maximum size of axes
that can be represented in this manner to 99, and that leading zero-padding
is not allowed.

\noindent \texttt{Nn\_j}: For $n > 1$ and for the case of arbitrary labels
along axis $n$, specifies the label string for the $j^{\mathrm{th}}$
point along the $n^{\mathrm{th}}$ axis.
The index $j$ shall begin with 1 and increase in integer steps up to the
size of the $n^{\mathrm{th}}$ axis, with one keyword per grid step.
These keywords do not belong to the FITS standard,
and can accommodate dimensionality up to 999.

For any given axis beyond the first (wavelength) axis, the
\texttt{redmonster} code will use information from the above keywords
to construct parameter-grid baselines.  The precedence
for establishing the baseline for each axis is
first for \texttt{CRPIXn, CRVALn, CDELTn}, then for
\texttt{PVn\_j}, then for \texttt{PSn\_j}, then
for \texttt{Nn\_j}, then for a one-based integer baseline
in the absence of a valid set of keywords of any of the preceding
specified types.

\section{Implementation}

Reader and writer routines for \texttt{ndArch} files conforming
to this proposed standard are implemented in the \texttt{redmonster.datamgr.io}
module as \texttt{read\_ndArch} and \texttt{write\_ndArch}.
(The location of these routines is subject to change
with future package reorganizations.)
Detailed documentation of these file-handling
routines can be found in their embedded docstrings.

A script called \texttt{test\_ndArch.py} is also provided,
which generates a file named \texttt{ndArch-TEST-v00.fits},
which in turn provides a unit test of the functioning of
the \texttt{read\_ndArch} routine.

\end{document}
